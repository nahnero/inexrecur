\documentclass{article}
\usepackage[spanish]{babel}
\usepackage[a4paper, top = 20mm, bottom=20mm, right=20mm, left=20mm] {geomet    ry}
\usepackage{graphicx}
\usepackage{amsmath}
\usepackage{wrapfig}
\usepackage{xcolor}
\usepackage{listings}
\usepackage{matlab-prettifier}
\usepackage{listingsutf8}
\usepackage[hidelinks]{hyperref}
\setlength{\parindent}{3mm}
\setlength{\parskip}{5mm}
\linespread{1.65}
\renewcommand{\figurename}{Fig.}

\title{\textbf{Trabajo Bioinform\'atica}\thanks{\href{https://www.upv.es/titulaciones/GIB/indexc.html}{Grado en Ingeniería Biomédica, Escuela Técnica Superior de Ingenieros Industriales, Valencia, España.}}}

\date{\today}
\author{
     \href{mailto:igamher@etsid.upv.es}{Ignacio Amat Hernández}
\and \href{mailto:ngellohe@etsii.upv.es}{Ángela LLopis Hernández}
\and \href{mailto:esbalho@etsii.upv.es}{Estrella Ballester Hoyo}
}

\lstset{
	inputencoding=utf8/latin1,
	basicstyle=\linespread{1.4}\selectfont,
	numbers=left,
	mathescape=true,
}

\newcommand{\tempcaption}{}

\newenvironment{code}[4]{
\begin{table}[h!]
\gdef\tempcaption{Traza de \textsc{InexRecur} con X = ``#1'', W = ``#2'', z = #3 en #4}
\centering
\begin{tabular}{|c|}
\hline }
{\\\hline
\end{tabular}
\caption{\protect\tempcaption}
\end{table} }

\newenvironment{codesnip}[1]{
\begin{table}[h!]
\gdef\tempcaption{#1}
\centering
\begin{tabular}{|c|}
\hline}
{\\\hline
\end{tabular}
\caption{\tempcaption}
\end{table}}

\newcommand{\C}{%
\raisebox{-.25\height}{\includegraphics[scale=.04]{../gallery/c.pdf}}{\hspace{0.5ex}}}
\newcommand{\R}{%
\raisebox{-.06\height}{\includegraphics[scale=.006]{../gallery/R.pdf}{\hspace{1ex}}}}
\newcommand{\python}{%
\raisebox{-.25\height}{\includegraphics[scale=.017]{../gallery/python3.pdf}}
}

\let\depth\relax
\begin{document}
\maketitle
\begin{figure}[h]
\centering
\includegraphics[width = 0.35\linewidth]{../gallery/googol.pdf}
\caption{Árbol de prefijos con intervalos SA de la palabra ``googol\$''.}
\end{figure}
\clearpage

\section{Materiales}

Los  lenguajes	que  hemos  comparado  son  \R,  \python  y  \C.    La
implementación en \R es la misma a la usada en	las  prácticas	de  la
asignatura.  Con la implementación proporcionada  hemos  dividido  dos
versiones, una que dibuja la traza y otra que calcula el resultado  de
forma limpia.  Todas las versiones del código escrito está  disponible
en el anejo.

El tiempo y la memoria la hemos medido de dos maneras. Primero hemos
evaluado a la función únicamente usando herramientas del lenguaje. En
\python hemos usado los paquetes
\lstinline[style=matlab-editor]{resource} y
\lstinline[style=matlab-editor]{time} para medir el tiempo y la
memoria. En \R hemos usado la utilidad
\lstinline[style=matlab-editor]{system.time} para el tiempo y la
libreria \lstinline[style=matlab-editor]{profmem} para perfilar el uso
de memoria. En \C{} hemos medido el tiempo usando la librería
\lstinline[style=matlab-editor]{time.h} y la memoria usando la
herramienta \lstinline[style=matlab-editor]{Valgrind}. Después hemos
medido el uso de recursos desde la línea de comandos, llamando a los
programas como scripts usando funcionalidades propias de cada
lenguaje. Con ello podemos evaluar el sobrecoste de cada una de las
tres plataformas usadas. Para ello hemos empleado la herramienta
\lstinline[style=matlab-editor]{bench}
% https://github.com/Gabriel439/bench
y el comando \lstinline[style=matlab-editor]{time}.
% https://www.gnu.org/software/time/

\begin{figure}[h!]
\centering
\includegraphics[width = 0.4\linewidth]{../gallery/ab.pdf}
\caption{Árbol de prefijos con intervalos SA de la palabra ``aaaabbbbb\$''.}
\end{figure}

\section{Resultados}

Hemos implementado las trazas de manera que son perfectamente
idénticas en \python y en \C. A continuación mostramos algunos
ejemplos comparativos para verificar que todos los programas se
comportan de forma igual. Posteriormente comparamos el uso de
tiempo y memoria.
\\
\begin{code}{googol\$}{gol}{0}{\C y \python}
\begin{lstlisting}
		   Mutation  i  z  k  l
 Deletion            [l]     1 -1  0  6
 Insertion           [g]     2 -1  1  2
 Substitution    [g] -> [l]  1 -1  1  2
 Insertion           [l]     2 -1  3  3
 Match               [l]     1  0  3  3
  Deletion           [o]     0 -1  3  3
  Insertion          [o]     1 -1  5  5
  Match              [o]     0  0  5  5
   Deletion          [g]    -1 -1  5  5
   Insertion         [g]     0 -1  1  1
   Match             [g]    -1  0  1  1
------------------- {1,1} -------------
 Insertion           [o]     2 -1  4  6
 Substitution    [o] -> [l]  1 -1  4  6
\end{lstlisting}
\end{code}
\phantom{}
\vfill
\begin{code}{googol\$}{gol}{0}{\R}
\begin{lstlisting}
|     INEXRECUR   -   by XAVI GABRI AITANA ALFREDO     |
-    D       [ l ]        1    -1    0    6
-    I       [ g ]        2    -1    1    2
-    S    [ g -> l ]    1    -1    1    2
-    I       [ l ]        2    -1    3    3
-    M       [ l ]        1    0    3    3
-    -    D       [ o ]        0    -1    3    3
-    -    I       [ o ]        1    -1    5    5
-    -    M       [ o ]        0    0    5    5
-    -    -    D       [ g ]        -1    -1    5    5
-    -    -    I       [ g ]        0    -1    1    1
-    -    -    M       [ g ]        -1    0    1    1
?-?-?-?-?-?-?-?-?-?-?-?-?-?-?-?-?-?-?-?-?-?-?-?-?-?-?-?->  [ c(1, 1) ]
-    I       [ o ]        2    -1    4    6
-    S    [ o -> l ]    1    -1    4    6
\end{lstlisting}
\end{code}
\vfill
\clearpage
\phantom{}
\vspace{1cm}
\begin{code}{googol\$}{goog}{0}{\C y \python}
\begin{lstlisting}
                   Mutation  i  z  k  l
 Deletion            [g]     2 -1  0  6
 Insertion           [g]     3 -1  1  2
 Match               [g]     2  0  1  2
  Deletion           [o]     1 -1  1  2
  Insertion          [o]     2 -1  4  4
  Match              [o]     1  0  4  4
   Deletion          [o]     0 -1  4  4
   Insertion         [o]     1 -1  6  6
   Match             [o]     0  0  6  6
 Insertion           [l]     3 -1  3  3
 Substitution    [l] -> [g]  2 -1  3  3
 Insertion           [o]     3 -1  4  6
 Substitution    [o] -> [g]  2 -1  4  6
\end{lstlisting}
\end{code}
\vfill
\begin{code}{googol\$}{goog}{0}{\R}
\begin{lstlisting}
|     INEXRECUR   -   by XAVI GABRI AITANA ALFREDO     |
-    D       [ g ]        2    -1    0    6
-    I       [ g ]        3    -1    1    2
-    M       [ g ]        2    0    1    2
-    -    D       [ o ]        1    -1    1    2
-    -    I       [ o ]        2    -1    4    4
-    -    M       [ o ]        1    0    4    4
-    -    -    D       [ o ]        0    -1    4    4
-    -    -    I       [ o ]        1    -1    6    6
-    -    -    M       [ o ]        0    0    6    6
-    I       [ l ]        3    -1    3    3
-    S    [ l -> g ]    2    -1    3    3
-    I       [ o ]        3    -1    4    6
-    S    [ o -> g ]    2    -1    4    6
\end{lstlisting}
\end{code}
\vfill
\clearpage
\phantom{}
\vfill
\begin{code}{googol\$}{gool}{1}{\C y \python}
\begin{lstlisting}
                   Mutation  i  z  k  l
 Deletion            [l]     2  0  1  6
  Deletion           [o]     1 -1  1  6
  Insertion          [g]     2 -1  1  2
  Substitution   [g] -> [o]  1 -1  1  2
  Insertion          [o]     2 -1  4  6
  Match              [o]     1  0  4  6
   Deletion          [o]     0 -1  4  6
   Insertion         [g]     1 -1  1  2
   Substitution  [g] -> [o]  0 -1  1  2
   Insertion         [o]     1 -1  6  6
   Match             [o]     0  0  6  6
    Deletion         [g]    -1 -1  6  6
    Insertion        [g]     0 -1  2  2
    Match            [g]    -1  0  2  2
------------------- {2,2} -------------
 Insertion           [g]     3  0  1  2
 Substitution    [g] -> [l]  2  0  1  2
  Deletion           [o]     1 -1  1  2
  Insertion          [o]     2 -1  4  4
  Match              [o]     1  0  4  4
   Deletion          [o]     0 -1  4  4
   Insertion         [o]     1 -1  6  6
   Match             [o]     0  0  6  6
    Deletion         [g]    -1 -1  6  6
    Insertion        [g]     0 -1  2  2
    Match            [g]    -1  0  2  2
------------------- {2,2} -------------
 Insertion           [o]     3  0  4  6
 Substitution    [o] -> [l]  2  0  4  6
  Deletion           [o]     1 -1  4  6
  Insertion          [g]     2 -1  1  2
  Substitution   [g] -> [o]  1 -1  1  2
  Insertion          [o]     2 -1  6  6
  Match              [o]     1  0  6  6
   Deletion          [o]     0 -1  6  6
   Insertion         [g]     1 -1  2  2
   Substitution  [g] -> [o]  0 -1  2  2
\end{lstlisting}
\end{code}
\vfill
\begin{code}{googol\$}{gool}{1}{\R}
\begin{lstlisting}
|     INEXRECUR   -   by XAVI GABRI AITANA ALFREDO     |
-    D       [ l ]        2    0    1    6
-    -    D       [ o ]        1    -1    1    6
-    -    I       [ g ]        2    -1    1    2
-    -    S    [ g -> o ]    1    -1    1    2
-    -    I       [ o ]        2    -1    4    6
-    -    M       [ o ]        1    0    4    6
-    -    -    D       [ o ]        0    -1    4    6
-    -    -    I       [ g ]        1    -1    1    2
-    -    -    S    [ g -> o ]    0    -1    1    2
-    -    -    I       [ o ]        1    -1    6    6
-    -    -    M       [ o ]        0    0    6    6
-    -    -    -    D       [ g ]        -1    -1    6    6
-    -    -    -    I       [ g ]        0    -1    2    2
-    -    -    -    M       [ g ]        -1    0    2    2
?-?-?-?-?-?-?-?-?-?-?-?-?-?-?-?-?-?-?-?-?-?-?-?-?-?-?-?->  [ c(2, 2) ]
-    I       [ g ]        3    0    1    2
-    S    [ g -> l ]    2    0    1    2
-    -    D       [ o ]        1    -1    1    2
-    -    I       [ o ]        2    -1    4    4
-    -    M       [ o ]        1    0    4    4
-    -    -    D       [ o ]        0    -1    4    4
-    -    -    I       [ o ]        1    -1    6    6
-    -    -    M       [ o ]        0    0    6    6
-    -    -    -    D       [ g ]        -1    -1    6    6
-    -    -    -    I       [ g ]        0    -1    2    2
-    -    -    -    M       [ g ]        -1    0    2    2
?-?-?-?-?-?-?-?-?-?-?-?-?-?-?-?-?-?-?-?-?-?-?-?-?-?-?-?->  [ c(2, 2) ]
-    I       [ o ]        3    0    4    6
-    S    [ o -> l ]    2    0    4    6
-    -    D       [ o ]        1    -1    4    6
-    -    I       [ g ]        2    -1    1    2
-    -    S    [ g -> o ]    1    -1    1    2
-    -    I       [ o ]        2    -1    6    6
-    -    M       [ o ]        1    0    6    6
-    -    -    D       [ o ]        0    -1    6    6
-    -    -    I       [ g ]        1    -1    2    2
-    -    -    S    [ g -> o ]    0    -1    2    2
\end{lstlisting}
\end{code}
\vfill

\clearpage

Se puede comprobar que las tres trazas contienen la misma información.
Procedemos ahora a la comparación de los tiempos medidos, primero las
mediciones tomadas desde el propio programa.
\begin{codesnip}{Tiempos de ``CPU''.}
\begin{lstlisting}
inexrecur_time.c
$\textcolor{blue}{\text{12.34} \mu s}$ from 10000 iterations.

inexrecur_time.py
real    sys   user
$\textcolor{blue}{\text{268.89} \mu s}$      0.37$\mu$s 268.16$\mu$s

inexrecur_time.R
user     system   elapsed
$\textcolor{blue}{\text{5422}\mu s}$         18$\mu$s      5443$\mu$s
\end{lstlisting}
\end{codesnip}

\vspace{-0.75cm}
En cada caso tomamos múltiples lecturas y hacemos la media. Como es de
esperar la solución compilada de \C{} es $\sim$20 veces más rápida al
script de \python y \R parece ser $\sim$450 veces más
lento que \C{} y $\sim$20 veces más lento que \python. Para
comparar el efecto del interpretador medimos ahora el tiempo
ejecutando desde la línea de comandos. LLamamos a los scripts usando
\texttt{\#!/bin/env Rscript} y \texttt{\#!/bin/env Python}.
\vspace{0.3cm}
\begin{codesnip}{Tiempo de ``pared'' según la utilidad \textit{bench}.}
\begin{lstlisting}
bench ./inexrecur_clean ./inexrecur_clean.py ./inexrecur_clean.R
benchmarking bench/./inexrecur_clean
time                 $\textcolor{blue}{\text{4.524 ms}}$           (4.496 ms .. 4.551 ms)
                     0.999 R$^2$     (0.999 R$^2$ .. 1.000 R$^2$)
mean                 4.522 ms    (4.499 ms .. 4.559 ms)
std dev              89.83 $\mu$s     (58.07 $\mu$s .. 133.1 $\mu$s)

benchmarking bench/./inexrecur_clean.py
time                 $\textcolor{blue}{\text{39.26 ms}}$           (39.12 ms .. 39.40 ms)
                     1.000 R$^2$    (1.000 R$^2$ .. 1.000 R$^2$)
mean                 39.30 ms   (39.18 ms .. 39.45 ms)
std dev              266.5 $\mu$s    (177.4 $\mu$s .. 388.7 $\mu$s)

benchmarking bench/./inexrecur_clean.R
time                 $\textcolor{blue}{\text{278.5 ms}}$           (273.6 ms .. 285.0 ms)
                     1.000 R$^2$    (1.000 R$^2$ .. 1.000 R$^2$)
mean                 280.6 ms   (279.1 ms .. 281.9 ms)
std dev              1.714 ms   (1.027 ms .. 2.517 ms)
variance introduced by outliers: 16% (moderately inflated)
\end{lstlisting}
\end{codesnip}

\vspace{-0.7cm}
En este caso \C{} es $\sim$10 veces más rápido a \python y $\sim$70
veces más rápido que \R, que continua siendo $\sim$7 veces más lento
que \python.

\clearpage
\begin{codesnip}{Memoria usada medido desde el script.}
\begin{lstlisting}
inexrecur_clean.c
==81469==     in use at exit: 18,588 bytes in 164 blocks
==81469==   total heap usage: 191 allocs, 27 frees, 24,894 bytes allocated

inexrecur_mem.py
6,631,424 bytes

inexrecur_mem.R
4,932,584 bytes
\end{lstlisting}
\end{codesnip}

La memoria usada es comparable en \R y \python cuando la medimos sin
tener en cuenta el entorno de ejecución, \C{} utiliza entorno a 200
veces menos memoria.

\vfill
\begin{codesnip}{Memoria usada medido desde fuera del script.}
\begin{lstlisting}
time ./inexrecur_clean
(5,5)
(1,1)
./inexrecur_clean
0.00s  user 0.00s system 44% cpu 0.004 total
max memory:                676 kB

time ./inexrecur_clean.py
(1, 1)
(5, 5)
./inexrecur_clean.py
0.03s  user 0.01s system 82% cpu 0.039 total
max memory:                6272 kB

time ./inexrecur_clean.R
[[1]]
[1] 5 5

[[2]]
[1] 1 1

./inexrecur_clean.R
0.23s  user 0.04s system 97% cpu 0.277 total
max memory:                67476 kB
\end{lstlisting}
\end{codesnip}

Cuando consideramos el entorno de ejecución interpretado al completo
\C{} se mantiene a la cabeza con la sorpresa de que \R emplea $\sim$10
veces más memoria que \python en este ejemplo en particular.

\clearpage

\phantom{}
\vspace{3cm}
\begin{figure}[h]
\centering
\includegraphics[width = 0.35\linewidth]{../gallery/bananas.pdf}
\caption{Árbol de prefijos con intervalos SA de la palabra ``bananas\$''.}
\end{figure}

\begin{figure}[h]
\centering
\includegraphics[width = \linewidth]{../gallery/gen.pdf}
\caption{Árbol de prefijos con intervalos SA de la palabra ``cgatagtcgggatggttgccg\$''.}
\end{figure}
\clearpage
\section{Anejo}

\lstinputlisting[
	frame=single,
	mathescape=false,
	basicstyle=\linespread{1}\ttfamily,
	caption={\protect\python inexrecur\_clean.py},
]{../inexrecur_clean.py}

\lstinputlisting[
	frame=single,
	mathescape=false,
	basicstyle=\linespread{1}\ttfamily,
	caption={\protect\R inexrecur\_clean.R},
]{../inexrecur_clean.R}

\lstinputlisting[
	firstline=27,
	lastline=85,
	frame=single,
	mathescape=false,
	basicstyle=\linespread{1}\ttfamily,
	caption={\protect\C inexrecur\_clean.c},
]{../inexrecur_clean.c}

\phantom{}
\vfill
\lstinputlisting[
	firstline=85,
	% lastline=115,
	firstnumber=59,
	frame=single,
	mathescape=false,
	basicstyle=\linespread{1}\ttfamily,
]{../inexrecur_clean.c}
\vfill

\end{document}
