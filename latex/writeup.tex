\documentclass{article}
\usepackage[spanish]{babel}
\usepackage[a4paper, top = 20mm, bottom=20mm, right=20mm, left=20mm] {geomet    ry}
\usepackage{graphicx}
\usepackage{amsmath}
\usepackage{wrapfig}
\usepackage{xcolor}
\usepackage{listings}
\usepackage{array}
\usepackage{matlab-prettifier}
\usepackage{listingsutf8}
\usepackage[utf8]{inputenc}
\usepackage[hidelinks]{hyperref}
\setlength{\parindent}{3mm}
\setlength{\parskip}{5mm}
\linespread{1.65}
\renewcommand{\figurename}{Fig.}

\title{\textbf{Implementación del algoritmo InexRecur en Python y
comparación con implementación en R.}\thanks{\href{https://www.upv.es/titulaciones/GIB/indexc.html}{Grado en Ingeniería Biomédica, Escuela Técnica Superior de Ingenieros Industriales, Valencia, España.}}}

\date{\today}
\author{
     \href{mailto:igamher@etsid.upv.es}{Ignacio Amat Hernández}
\and \href{mailto:ngellohe@etsii.upv.es}{Ángela LLopis Hernández}
\and \href{mailto:esbalho@etsii.upv.es}{Estrella Ballester Hoyo}
}

\lstset{
	inputencoding=utf8/latin1,
	basicstyle=\linespread{1.4}\selectfont,
	numbers=left,
	mathescape=true,
}

\newcommand{\tempcaption}{}
\usepackage{tocloft}
\renewcommand{\cftsecleader}{\cftdotfill{\cftdotsep}}
\newenvironment{code}[4]{
\begin{table}[h!]
\gdef\tempcaption{Traza de \textsc{InexRecur} con X = ``#1'', W = ``#2'', z = #3 en #4}
\centering
\begin{tabular}{|c|}
\hline }
{\\\hline
\end{tabular}
\caption{\tempcaption}
\end{table} }

\newenvironment{codesnip}[1]{
\begin{table}[h!]
\gdef\tempcaption{#1}
\centering
\begin{tabular}{|c|}
\hline}
{\\\hline
\end{tabular}
\caption{\tempcaption}
\end{table}}

% \newcommand{\C}{%
% \raisebox{-.25\height}{\includegraphics[scale=.04]{../gallery/c.pdf}}{\hspace{0.5ex}}}
% \newcommand{\R}{%
% \raisebox{-.06\height}{\includegraphics[scale=.006]{../gallery/R.pdf}{\hspace{1ex}}}}
% \newcommand{\python}{%
% \raisebox{-.25\height}{\includegraphics[scale=.017]{../gallery/python3.pdf}}
% }

\newcommand{\C}{C }
\newcommand{\R}{R }
\newcommand{\python}{Python }

\let\depth\relax
\newcolumntype{P}[1]{>{\centering\arraybackslash}p{#1}}
\begin{document}
\maketitle
\begin{figure}[h]
\centering
\includegraphics[width = 0.35\linewidth]{../gallery/googol.pdf}
\caption{Árbol de prefijos con intervalos SA de la palabra ``googol\$''.}
\label{fig:arbol1}
\end{figure}

\clearpage

\tableofcontents
\listoffigures
\listoftables
\lstlistoflistings

\section*{Resumen}

Debido a la aparición de nuevas técnicas de secuenciación masiva (NGS)
en la bioinformática, existen actualmente numerosas herramientas  para
la secuenciación. Algunas de ellas han sido diseñadas para resolver el
problema de  alineamiento,  entre  ellas  el  Burrows-Wheeler  Aligner
(BWA), un alineador diseñado para alinear de secuencias a partir de un
genoma	de  referencia \cite{gimenez_2016}.   BWA  se	basa   en   la
transformada de Burrows-Wheeler (BWT) y tiene la capacidad de realizar
búsquedas inexactas (permitiendo huecos y errores) de forma  recursiva
mediante el método InexRecur \cite{li_durbin_2009}.

Hemos programado el algoritmo de recursión InexRecur en Python para el
alineamiento de secuencias cortas con secuencias de referencia de gran
tamaño	y  hemos  realizado  diversas  pruebas	 comparando   con   la
implementación del mismo algoritmo en R.  Hemos obtenido las trazas en
Python, R y C, los tiempos de CPU, la memoria requerida y los  árboles
de prefijos.

Los resultados obtenidos muestran que la implementación de la  función
de búsqueda recursiva inexacta en Python proporciona un  menor	tiempo
de procesamiento y menor uso de memoria, además de poseer una sintaxis
más   intuitiva   para	 el   usuario	en    comparación    con    R.

\section{Introducción}

El campo de la bioinformática es una rama de la informática  combinada
con la biología, la química, la estadística y muchos otros ámbitos que
trata de recopilar, almacenar, analizar,  manipular,  etc.   datos  de
carácter biológico con el fin de comprender el funcionamiento  de  los
organismos vivos, su origen y poder hallar así principios  universales
relativos a las ciencias de la vida \cite{gimenez_2016}.  Es un ámbito
de gran interés para la industria científica pero que ha de manejar un
gran  número  de  datos  (normalmente  una  máquina  de  secuenciación
Illumina/Solexa puede producir de 50 a 200 millones de lecturas  32  a
100 pares de bases en una sola ejecución \cite{li_durbin_2009}) y  con
ello la bioinformática ha experimentado en los últimos años  una  gran
evolución; cada vez secuenciar el genoma es más asequible  y  esto  es
gracias  a   las   denominadas	 New   Generation   Sequencing	 (NGS)
\cite{gimenez_2016}, nuevas técnicas más  potentes  y  menos  costosas
computacionalmente. Dichas técnicas emplean algoritmos de alineamiento
que  permiten  llevar  a  cabo	una  secuenciación  masiva   del   ADN
\cite{li_durbin_2009}  con  el	fin  de  conocer  sus  secuencias   de
nucleótidos y poder compararlas  entre	individuos  de	las  mismas  o
diferentes especies, así como hallar el origen evolutivo, la ortología
u homología genética... .Existen diversos relativos a la secuenciación
genómica  que  las   NGS   tratan   de	 abordar,   entre   ellos   la
resecuenciación, el mapeo o alineación y el ensamblado “de  novo”  del
genoma\cite{gimenez_2016}.

En el presente trabajo nos centramos en el problema de	la  alineación
que  trata  de,  a  partir  de	lecturas  de  un  genoma  desconocido,
compararlas con un genoma de referencia conocido.   Para  dicha  tarea
existen numerosas técnicas que permiten resolver alineamientos pero no
todas ellas funcionan de la misma forma.  Entre  los  alineadores  más
empleados en la actualidad encontramos Bowtie  y  BWA.	 Nosotros  nos
centraremos en este último.

Burrows-Wheeler  Aligner  (BWA)  es  un   alineador   basado   en   la
transformada de Burrows-Wheeler (BWT) que  permite,  a	partir	de  un
genoma de referencia,  compararlo  con	secuencias  cortas  realizando
búsquedas recursivas.  Además, a diferencia de otros alineadores  como
Bowtie, BWA permite errores y gaps (huecos) en las secuencias, que son
mutaciones genéticas muy comunes \cite{gimenez_2016}.  De  esta  forma
BWA nos ofrece la  posibilidad	de  realizar  búsquedas  exactas  (sin
errores ni gaps) o  inexactas(con  errores  y  gaps)\cite{gomez_2020}.
En concreto, InexRecur es el algoritmo recursivo empleado por BWA para
la búsqueda inexacta.  Esta función es de gran utilidad a la  hora  de
alinear secuencias  de	genoma	muy  largas,  puesto  que  existe  una
gran probabilidad de haberse producido	errores  y  gaps.   En	cambio
Bowtie	se  suele  emplear  más  a  menudo  en	secuencias  de	 menor
tamaño	 en   las   que   se   busca   una    alta    correspondencia.

Por otro lado, en cuanto a lenguajes de programación existen  un  gran
número en el mercado.  Y en concreto, para el análisis de datos Python
y R son  los  más  empleados  para  ello.   En	el  presente  artículo
presentaremos el algoritmo en InexRecur en el lenguaje de programación
Python	 y   la   compararemos	 con   su   implementación    en    R.

\section{Métodos}
\subsection{Precálculo}
\subsubsection{Árbol de prefijos}

El árbol de prefijos es una estructura con forma de árbol  obtenida  a
partir de una cadena de texto de referencia X en la que cada  carácter
es referenciado por una arista.   De  esta  manera,  concatenando  los
caracteres de una arista de un nodo a la raíz se obtiene un  substring
único de X o cadena representada por el nodo.  Cabe  destacar  que  el
árbol de prefijos de X es igual al árbol de  sufijos  de  X  invertida
(X’) Esta herramienta facilita la búsqueda de, dado  un  substring  W,
saber si pertenece a X,  pues  solo  habrá  que  buscar  el  nodo  que
representa a W.  Podemos observar un ejemplo de un árbol  de  prefijos
en la \hyperref[fig:arbol1]{{Fig.}~\ref*{fig:arbol1},}
\hyperref[fig:arbol2]{\ref*{fig:arbol2},}
\hyperref[fig:arbol3]{\ref*{fig:arbol3},}
\hyperref[fig:arbol4]{\ref*{fig:arbol4}}.

\subsubsection{Vector de sufijos (B[i]) para la cadena de referencia X
}


El primer paso de  la  búsqueda  inexacta  en  Burrows-Wheeler	es  la
creación del vector de sufijos	B[i]  dada  una  cadena  de  texto  de
referencia X.  B[i] será una cadena de texto con los mismos caracteres
que X pero ordenada lexicográficamente \cite{belda_2020}. En la Figura
1 mostramos un ejemplo sencillo de la contrucción de B[i] a partir  de
un X de referencia.  Hemos de recalcar que en la práctica, X sería una
cadena de texto con un gran número de  caracteres  (nucleótidos)  y  W
correspondería	a  una	secuencia  que	 queramos   alinear   con   X.
\begin{figure}[h]
\centering
\includegraphics[width = 0.5\linewidth]{../gallery/dia1.pdf}
\caption{Ejemplo de la creación del vector de sufijos.}
\label{fig:dia1}
\end{figure}

\subsubsection{Vector C(a)}

El vector C(a) en Burrows-Wheeler se obtiene para observar si W es un
substring perteneciente a X, y representa el número de caracteres en
X[0, n-2] lexicográficamente inferiores a a. En la Figura 2 mostramos
un ejemplo del vector C(a).

\subsubsection{Vector O(a,i)}

Otro vector implicado en  BWA  es  O(a,i)  que	indica	el  número  de
ocurrencias de un símbolo a en B[0,i].	Además, en InexRecur se  puede
realizar búsquedas inexactas de manera que, dado un W no perteneciente
a X pero con un número de errores  menores  a  un  umbral  z,  podemos
encontrar secuencias similares en el árbol de prefijos. En la Figura 2
podemos observar un ejemplo de O(a,i).


\begin{table}[h!]
\centering
\begin{tabular}{|P{3cm}|P{3cm}|}
	\hline
	X = `googol\$' & B[i] = `lo\$oogg' \\\hline
	W = `lol' & C = (3, 2, 3) \\
\end{tabular}
\begin{tabular}{|P{3cm}|P{3cm}|P{3cm}|}
	\hline
\multicolumn{3}{|c|}{O(a,i)} \\ \hline
i & l & o \\ \hline
0 & 1 & 0 \\
1 & 1 & 1 \\
2 & 1 & 1 \\
3 & 1 & 2 \\
4 & 1 & 3 \\
5 & 1 & 3 \\
6 & 1 & 3 \\ \hline
\end{tabular}
\caption{Obtención de los vectores C(a) y O(a,i) dado  W y X.}
\label{table:tab1}
\end{table}


\subsubsection{Vector de sufijos B’[i] para la cadena de referencia
invertida X’}

Siguiendo con el ejemplo de las Figuras 1 y  2,  podríamos  obtener  a
partir de X=`googol\$’ su cadena invertida X=`\$logoog’.  Y el cálculo
de B’[i] de X’ se realizará de forma análoga al de B[i] para un string
X.  Como hemos mencionado anteriormente, el árbol de prefijos de X  es
equivalente al árbol de sufijos de X’.

\subsubsection{Vector O’(a,i) para B’[i]}

Una vez obtenido B’[i] calculamos su función correspondiente O’(a,i)
(Ejemplo en la \hyperref[table:tab1]{Cuadro~\ref*{table:tab1}}).
Este es el último paso del precálculo, a partir del cual ya podremos
llevar a cabo la búsqueda inexacta.


\subsection{Búsqueda inexacta}
\vspace{-0.25cm}
\subsubsection{Función CalculateD}
\vspace{-0.25cm}

Previamente a comenzar con la función InexRecur deberemos  implementar
la función CalculateD para obtener el  vector  D.   Este  vector  hace
referencia al número de fallos realizados hasta el momento, a variable
z.

El parámetro de entrada a  la  función	será  el  substring  W,  y  el
parámetro de salida será el propio vector D.  La lógica llevada a cabo
en la  función	será  la  mostrada  en	la  Figura  3.	 El  algoritmo
establecerá unos valores para el umbral z y para el  primer  parámetro
de W, j, e irá recorriendo en un bucle la subcadena W  comprobando  si
pertenece o no a X.  Finalmente asignará el valor de z a D y devolverá
este último vector.

\begin{codesnip}{Algoritmo de la función \textsc{CalculateD}.\cite{li_durbin_2009}}
\begin{lstlisting}[mathescape=true, language = C]
$\textsc{CalculateD(W)}$
   $z\leftarrow 0$
   $j\leftarrow 0$
   for $i=0$ to $\left|W\right|-1$ do
   if W[j,i] is not a substring of X $\textbf{then}$
     $z\leftarrow z+1$
     $j\leftarrow i+1$
   $D(i)\leftarrow z$
\end{lstlisting}
\end{codesnip}

\vspace{-0.5cm}
\subsubsection{Función de búsqueda inexacta recursiva InexRecur}

Esta función es la encargada de realizar la búsqueda inexacta de forma
recursiva.   Tendrá  en  cuenta  los  diferentes  tipos  posibles   de
operaciones (match, sustituciones,  deleciones	e  inserciones).   Los
parámetros de entrada de InexRecur serán el substring W, el número  de
errores   z,   y   las	 variables   de   recursión   i,   k   y    l.

\begin{codesnip}{Algoritmo de la función \textsc{InexRecur}.\cite{li_durbin_2009}}
\begin{lstlisting}[mathescape=true, language = C]
$\textsc{InexRecur}(W,i,z,k,l)$
   if $z<D(i)$ then
   	return $\emptyset$
   if $i<0$ then
   	return $\{[k,l]\}$
	I$\leftarrow \emptyset$
   I $\leftarrow$ I $\cup$ $\textsc{InexRecur}(W,i-1,z-1,k,l)$
   for each b$\in${A,C,G,T} do
   	$k\leftarrow C(b)+O(b,k-1)+1$
   	$l\leftarrow C(b)+O(b,l)$
   	if k$\leq$l then
   	    I $\leftarrow$ I $\cup$ $\textsc{InexRecur}(W,i,z-1,k,l)$
   	    if $b=W[i]$ then
   		I $\leftarrow$ I $\cup$ $\textsc{InexRecur}(W,i-1,z,k,l)$
   	    else
   		I $\leftarrow$ I $\cup$ $\textsc{InexRecur}(W,i-1,z-1,k,l)$
   return I
\end{lstlisting}
\end{codesnip}

Vemos pues que para cada posible tipo de operaciones el algoritmo
llevará a cabo una acción concreta.

En el caso de que se  haya  producido  una  deleción  (borrado	de  un
carácter o nucleótido en dicha	posición)  se  llamará	a  la  función
recursiva disminuyendo una unidad la variable de posición y el	número
de errores (i, z).   Si  en  cambio  se  ha  observado	una  inserción
(adición de un carácter o nucleótido en la secuencia) se llamará a  la
función InexRecur y se disminuirá en un tanto el número de errores  z.

En el caso de que se haya producido  un  match	(coincidencia)	o  una
mutación de  tipo  sustitución	se  volverá  a	llamar	a  la  función
InexRecur pero disminuyendo en una unidad la  variable  i.

Y por último, si se ha producido una  sustitución,  se	llamará  a  la
función InexRecur disminuyendo en un tanto la variable de  posición  y
el número de errores (i, z), como ocurría en el caso de una  deleción.

\subsubsection{Función InexSearch}

Esta función será la encargada de llamar a la función InexRecur,
pasándole sus correspondientes parámetros de entrada W, i, z, k y l.
Las variables de entrada a la función InexSearch serán por tanto el
substring W y el umbral z.  Podemos ver la lógica de esta función en
la Figura 5.
\\
\begin{codesnip}{Algoritmo de la función \textsc{InexSearch}.\cite{li_durbin_2009}}
\begin{lstlisting}[mathescape=true, language = C]
$\textsc{InexactSearch}(W,z)$
	$\textsc{CalculateD}(W)$
	return $\textsc{InexRecur}(W,|W|-1,z,1,|X|-1)$
\end{lstlisting}
\end{codesnip}

\vspace{-1cm}
\section{Materiales}

Los lenguajes que hemos comparado son R, Python y C. La implementación
en \R es la misma a la usada en las prácticas de la asignatura. Con la
implementación proporcionada hemos dividido  dos  versiones,  una  que
dibuja la traza y otra que calcula el resultado de forma limpia. Todas
las  versiones	del  código  escrito  está  disponible	en  el	anejo.

El tiempo y la memoria la hemos medido de dos maneras.	Primero  hemos
evaluado a la función únicamente usando herramientas del lenguaje.  En
\python hemos usado los paquetes
\lstinline[style=matlab-editor]{resource}\cite{resource_2020} y
\lstinline[style=matlab-editor]{time}\cite{time_2020}  para  medir  el
tiempo	 y   la   memoria.    En   \R	hemos	usado	la    utilidad
\lstinline[style=matlab-editor]{system.time}\cite{system.time_documentation}
para el tiempo y la libreria
\lstinline[style=matlab-editor]{profmem}\cite{bengtsson} para perfilar
el uso de memoria.  En \C{} hemos medido el tiempo usando la  librería
\lstinline[style=matlab-editor]{time.h}\cite{timeh}   y   la   memoria
usando la herramienta
\lstinline[style=matlab-editor]{Valgrind}\cite{Valgrind10}.    Después
hemos medido el uso de recursos desde la línea de comandos, llamando a
los programas como scripts  usando  funcionalidades  propias  de  cada
lenguaje.  Con ello podemos evaluar el sobrecoste de cada una  de  las
tres plataformas usadas.  Para	ello  hemos  empleado  la  herramienta
\lstinline[style=matlab-editor]{bench}\cite{GitHubGa29} y  el  comando
\lstinline[style=matlab-editor]{time}\cite{timeh}.

\section{Resultados}

Hemos implementado las trazas de manera que son perfectamente
idénticas en \python y en \C. A continuación mostramos algunos
ejemplos comparativos para verificar que todos los programas se
comportan de forma igual. Posteriormente comparamos el uso de
tiempo y memoria.
\\
\begin{code}{googol\$}{gol}{0}{\C y \python}
\begin{lstlisting}
		   Mutation  i  z  k  l
 Deletion            [l]     1 -1  0  6
 Insertion           [g]     2 -1  1  2
 Substitution    [g] -> [l]  1 -1  1  2
 Insertion           [l]     2 -1  3  3
 Match               [l]     1  0  3  3
  Deletion           [o]     0 -1  3  3
  Insertion          [o]     1 -1  5  5
  Match              [o]     0  0  5  5
   Deletion          [g]    -1 -1  5  5
   Insertion         [g]     0 -1  1  1
   Match             [g]    -1  0  1  1
------------------- {1,1} -------------
 Insertion           [o]     2 -1  4  6
 Substitution    [o] -> [l]  1 -1  4  6
\end{lstlisting}
\end{code}
\phantom{}
\vfill
\begin{code}{googol\$}{gol}{0}{\R}
\begin{lstlisting}
|     INEXRECUR   -   by XAVI GABRI AITANA ALFREDO     |
-    D       [ l ]        1    -1    0    6
-    I       [ g ]        2    -1    1    2
-    S    [ g -> l ]    1    -1    1    2
-    I       [ l ]        2    -1    3    3
-    M       [ l ]        1    0    3    3
-    -    D       [ o ]        0    -1    3    3
-    -    I       [ o ]        1    -1    5    5
-    -    M       [ o ]        0    0    5    5
-    -    -    D       [ g ]        -1    -1    5    5
-    -    -    I       [ g ]        0    -1    1    1
-    -    -    M       [ g ]        -1    0    1    1
?-?-?-?-?-?-?-?-?-?-?-?-?-?-?-?-?-?-?-?-?-?-?-?-?-?-?-?->  [ c(1, 1) ]
-    I       [ o ]        2    -1    4    6
-    S    [ o -> l ]    1    -1    4    6
\end{lstlisting}
\end{code}
\vfill
\clearpage
\phantom{}
\vspace{1cm}
\begin{code}{googol\$}{goog}{0}{\C y \python}
\begin{lstlisting}
                   Mutation  i  z  k  l
 Deletion            [g]     2 -1  0  6
 Insertion           [g]     3 -1  1  2
 Match               [g]     2  0  1  2
  Deletion           [o]     1 -1  1  2
  Insertion          [o]     2 -1  4  4
  Match              [o]     1  0  4  4
   Deletion          [o]     0 -1  4  4
   Insertion         [o]     1 -1  6  6
   Match             [o]     0  0  6  6
 Insertion           [l]     3 -1  3  3
 Substitution    [l] -> [g]  2 -1  3  3
 Insertion           [o]     3 -1  4  6
 Substitution    [o] -> [g]  2 -1  4  6
\end{lstlisting}
\end{code}
\vfill
\begin{code}{googol\$}{goog}{0}{\R}
\begin{lstlisting}
|     INEXRECUR   -   by XAVI GABRI AITANA ALFREDO     |
-    D       [ g ]        2    -1    0    6
-    I       [ g ]        3    -1    1    2
-    M       [ g ]        2    0    1    2
-    -    D       [ o ]        1    -1    1    2
-    -    I       [ o ]        2    -1    4    4
-    -    M       [ o ]        1    0    4    4
-    -    -    D       [ o ]        0    -1    4    4
-    -    -    I       [ o ]        1    -1    6    6
-    -    -    M       [ o ]        0    0    6    6
-    I       [ l ]        3    -1    3    3
-    S    [ l -> g ]    2    -1    3    3
-    I       [ o ]        3    -1    4    6
-    S    [ o -> g ]    2    -1    4    6
\end{lstlisting}
\end{code}
\vfill
\clearpage
\phantom{}
\vfill
\begin{code}{googol\$}{gool}{1}{\C y \python}
\begin{lstlisting}
                   Mutation  i  z  k  l
 Deletion            [l]     2  0  1  6
  Deletion           [o]     1 -1  1  6
  Insertion          [g]     2 -1  1  2
  Substitution   [g] -> [o]  1 -1  1  2
  Insertion          [o]     2 -1  4  6
  Match              [o]     1  0  4  6
   Deletion          [o]     0 -1  4  6
   Insertion         [g]     1 -1  1  2
   Substitution  [g] -> [o]  0 -1  1  2
   Insertion         [o]     1 -1  6  6
   Match             [o]     0  0  6  6
    Deletion         [g]    -1 -1  6  6
    Insertion        [g]     0 -1  2  2
    Match            [g]    -1  0  2  2
------------------- {2,2} -------------
 Insertion           [g]     3  0  1  2
 Substitution    [g] -> [l]  2  0  1  2
  Deletion           [o]     1 -1  1  2
  Insertion          [o]     2 -1  4  4
  Match              [o]     1  0  4  4
   Deletion          [o]     0 -1  4  4
   Insertion         [o]     1 -1  6  6
   Match             [o]     0  0  6  6
    Deletion         [g]    -1 -1  6  6
    Insertion        [g]     0 -1  2  2
    Match            [g]    -1  0  2  2
------------------- {2,2} -------------
 Insertion           [o]     3  0  4  6
 Substitution    [o] -> [l]  2  0  4  6
  Deletion           [o]     1 -1  4  6
  Insertion          [g]     2 -1  1  2
  Substitution   [g] -> [o]  1 -1  1  2
  Insertion          [o]     2 -1  6  6
  Match              [o]     1  0  6  6
   Deletion          [o]     0 -1  6  6
   Insertion         [g]     1 -1  2  2
   Substitution  [g] -> [o]  0 -1  2  2
\end{lstlisting}
\end{code}
\vfill
\begin{code}{googol\$}{gool}{1}{\R}
\begin{lstlisting}
|     INEXRECUR   -   by XAVI GABRI AITANA ALFREDO     |
-    D       [ l ]        2    0    1    6
-    -    D       [ o ]        1    -1    1    6
-    -    I       [ g ]        2    -1    1    2
-    -    S    [ g -> o ]    1    -1    1    2
-    -    I       [ o ]        2    -1    4    6
-    -    M       [ o ]        1    0    4    6
-    -    -    D       [ o ]        0    -1    4    6
-    -    -    I       [ g ]        1    -1    1    2
-    -    -    S    [ g -> o ]    0    -1    1    2
-    -    -    I       [ o ]        1    -1    6    6
-    -    -    M       [ o ]        0    0    6    6
-    -    -    -    D       [ g ]        -1    -1    6    6
-    -    -    -    I       [ g ]        0    -1    2    2
-    -    -    -    M       [ g ]        -1    0    2    2
?-?-?-?-?-?-?-?-?-?-?-?-?-?-?-?-?-?-?-?-?-?-?-?-?-?-?-?->  [ c(2, 2) ]
-    I       [ g ]        3    0    1    2
-    S    [ g -> l ]    2    0    1    2
-    -    D       [ o ]        1    -1    1    2
-    -    I       [ o ]        2    -1    4    4
-    -    M       [ o ]        1    0    4    4
-    -    -    D       [ o ]        0    -1    4    4
-    -    -    I       [ o ]        1    -1    6    6
-    -    -    M       [ o ]        0    0    6    6
-    -    -    -    D       [ g ]        -1    -1    6    6
-    -    -    -    I       [ g ]        0    -1    2    2
-    -    -    -    M       [ g ]        -1    0    2    2
?-?-?-?-?-?-?-?-?-?-?-?-?-?-?-?-?-?-?-?-?-?-?-?-?-?-?-?->  [ c(2, 2) ]
-    I       [ o ]        3    0    4    6
-    S    [ o -> l ]    2    0    4    6
-    -    D       [ o ]        1    -1    4    6
-    -    I       [ g ]        2    -1    1    2
-    -    S    [ g -> o ]    1    -1    1    2
-    -    I       [ o ]        2    -1    6    6
-    -    M       [ o ]        1    0    6    6
-    -    -    D       [ o ]        0    -1    6    6
-    -    -    I       [ g ]        1    -1    2    2
-    -    -    S    [ g -> o ]    0    -1    2    2
\end{lstlisting}
\end{code}
\vfill

\clearpage

Se puede comprobar que las tres trazas contienen la misma información.
Procedemos ahora a la comparación de los tiempos medidos, primero  las
mediciones tomadas desde el propio programa.
\begin{codesnip}{Tiempos de ``CPU''.}
\begin{lstlisting}
inexrecur_time.c
$\textcolor{blue}{\text{12.34} \mu s}$ from 10000 iterations.

inexrecur_time.py
real    sys   user
$\textcolor{blue}{\text{268.89} \mu s}$      0.37$\mu$s 268.16$\mu$s

inexrecur_time.R
user     system   elapsed
$\textcolor{blue}{\text{5422}\mu s}$         18$\mu$s      5443$\mu$s
\end{lstlisting}
\end{codesnip}

\vspace{-0.75cm}
En cada caso tomamos múltiples lecturas y hacemos la media. Como es de
esperar la solución compilada de \C{} es $\sim$20 veces más rápida  al
script de \python y \R parece ser $\sim$450 veces más lento que \C{} y
$\sim$20 veces más lento que \python.  Para  comparar  el  efecto  del
interpretador medimos ahora el tiempo ejecutando  desde  la  línea  de
comandos. LLamamos a los scripts usando \texttt{\#!/bin/env Rscript} y
\texttt{\#!/bin/env Python}.
\vspace{0.3cm}
\begin{codesnip}{Tiempo de ``pared'' según la utilidad \textit{bench}.}
\begin{lstlisting}
bench ./inexrecur_clean ./inexrecur_clean.py ./inexrecur_clean.R
benchmarking bench/./inexrecur_clean
time                 $\textcolor{blue}{\text{4.524 ms}}$           (4.496 ms .. 4.551 ms)
                     0.999 R$^2$     (0.999 R$^2$ .. 1.000 R$^2$)
mean                 4.522 ms    (4.499 ms .. 4.559 ms)
std dev              89.83 $\mu$s     (58.07 $\mu$s .. 133.1 $\mu$s)

benchmarking bench/./inexrecur_clean.py
time                 $\textcolor{blue}{\text{39.26 ms}}$           (39.12 ms .. 39.40 ms)
                     1.000 R$^2$    (1.000 R$^2$ .. 1.000 R$^2$)
mean                 39.30 ms   (39.18 ms .. 39.45 ms)
std dev              266.5 $\mu$s    (177.4 $\mu$s .. 388.7 $\mu$s)

benchmarking bench/./inexrecur_clean.R
time                 $\textcolor{blue}{\text{278.5 ms}}$           (273.6 ms .. 285.0 ms)
                     1.000 R$^2$    (1.000 R$^2$ .. 1.000 R$^2$)
mean                 280.6 ms   (279.1 ms .. 281.9 ms)
std dev              1.714 ms   (1.027 ms .. 2.517 ms)
variance introduced by outliers: 16% (moderately inflated)
\end{lstlisting}
\end{codesnip}

\vspace{-0.7cm}
En este caso \C{} es $\sim$10 veces más rápido a \python y $\sim$70
veces más rápido que \R, que continua siendo $\sim$7 veces más lento
que \python.

\clearpage
\begin{codesnip}{Memoria usada medido desde el script.}
\begin{lstlisting}
inexrecur_clean.c
==81469==     in use at exit: 18,588 bytes in 164 blocks
==81469==   total heap usage: 191 allocs, 27 frees, 24,894 bytes allocated

inexrecur_mem.py
6,631,424 bytes

inexrecur_mem.R
4,932,584 bytes
\end{lstlisting}
\end{codesnip}

La memoria usada es comparable en \R y \python cuando la medimos sin
tener en cuenta el entorno de ejecución, \C{} utiliza entorno a 200
veces menos memoria.

\vfill
\begin{codesnip}{Memoria usada medido desde fuera del script.}
\begin{lstlisting}
time ./inexrecur_clean
(5,5)
(1,1)
./inexrecur_clean
0.00s  user 0.00s system 44% cpu 0.004 total
max memory:                676 kB

time ./inexrecur_clean.py
(1, 1)
(5, 5)
./inexrecur_clean.py
0.03s  user 0.01s system 82% cpu 0.039 total
max memory:                6272 kB

time ./inexrecur_clean.R
[[1]]
[1] 5 5

[[2]]
[1] 1 1

./inexrecur_clean.R
0.23s  user 0.04s system 97% cpu 0.277 total
max memory:                67476 kB
\end{lstlisting}
\end{codesnip}

Cuando consideramos el entorno de ejecución interpretado al completo C
se mantiene a la cabeza con la sorpresa de que R emplea $\sim$10 veces
más memoria que \python en este ejemplo en particular.

\section{Discusión}
\vspace{-0.25cm}

Como se ha visto en el apartado de Resultados, hemos  implementado  el
algoritmo BWA  en  los	lenguajes  R,  Python  y  C;  hemos  utilizado
distintos criterios para evaluar el rendimiento de  éstos  y  realizar
una comparación de su funcionamiento.  Dichos criterios son el	tiempo
de  procesamiento  y  la  memoria  usada.   En	el  caso  del  tiempo,
centrándonos en los  dos  lenguajes  que  queremos  comparar  en  este
artículo, el que  ofrece  un  tiempo  menor  medido  desde  el	propio
programa a la hora de alinear las secuencias es Python, mientras que R
resulta ser más  lento	en  comparación  con  éste.   Por  otro  lado,
evaluando la memoria, Python  ocupa  aproximadamente  10  veces  menos
memoria que R midiendo desde fuera del script.

Python es un lenguaje con mensaje interactivo,	permite  la  escritura
dinámica y administra la memoria automáticamente.  Tiene una  sintaxis
muy clara e intuitiva, y utiliza márgenes para hacer  los  bloques  de
declaraciones.	Además, se trata de un software libre por  lo  que  se
puede acceder a él fácilmente y desde todos los  sistemas  operativos.
\cite{fangohr_2004}


R se trata de un lenguaje de  programación  con  enfoque  de  análisis
estadístico, no se suele usar para implementar programas  largos  como
se hace con otros lenguajes.  Permite combinar programación imperativa
con funcional, además de incluir  programación	orientada  a  objetos.
Asimismo, es uno de los  lenguajes  más  utilizados  en  investigación
científica. \cite{bellosta_2018,kleiber_zeileis_2008}

Así pues, basándonos en los ejemplos que hemos ejecutado,  parece  ser
que en Python es más sencillo implementar el algoritmo InexRecur  (ver
Anexo 1) y además presenta un mejor valor en los  criterios  evaluados
anteriormente.	Por lo	tanto,	dado  que  alinear  secuencias	es  un
procedimiento que ya de por sí consume mucho tiempo,  podríamos  decir
que a la  hora	de  mapear  una  secuencia  corta  con	un  genoma  de
referencia sería más rápido  y	eficaz	utilizar  el  lenguaje	Python
\cite{amigo_2015}.  Sin embargo, hay que tener en cuenta que como este
algoritmo es computacionalmente costoso no será conveniente utilizarlo
para alinear secuencias cortas con,  por  ejemplo,  el	genoma	humano
completo,  ya  que  daría  unos  tiempos  de  procedimiento  demasiado
elevados \cite{Houtgast2015}.  De la misma forma, también  se  comenta
este mismo inconveniente en el	artículo  de  Heng  Li	et  al.,  2009
\cite{li_durbin_2009} donde se expone que el rendimiento del alineador
BWA se reduce cuando son lecturas largas, por lo que aconsejan	que  a
la hora de trabajar con secuencias largas  se  divida  la  lectura  en
varios fragmentos cortos, alinearlos  por  separado  y	posteriormente
unir las alineaciones parciales en una global.

\vspace{-1cm}
\section{Conclusiones}
\vspace{-0.25cm}

Hemos propuesto una implementación del algoritmo InexRecur  en	Python
para  alineamientos  inexactos	de  fragmentos	 de   secuencias   con
secuencias largas. Lo implementamos de forma correcta y comprobamos su
funcionamiento con  varios  ejemplos  con  la  traza  que  se  nos  ha
proporcionado.	Se comprueba que la programación de este algoritmo  en
Python proporciona un gran  potencial  para  disminuir	el  tiempo  de
procesamiento y la memoria total  utilizada,  además  de  ofrecer  una
sintaxis  más  clara  en  comparación  con  la	implementación	en  R.


\clearpage
\phantom{}
\vfill
\begin{figure}[h!]
\centering
\includegraphics[width = 0.6\linewidth]{../gallery/ab.pdf}
\caption{Árbol de prefijos con intervalos SA de la palabra ``aaaabbbbb\$''.}
\label{fig:arbol2}
\end{figure}
\vfill

\clearpage

\phantom{}
\vspace{3cm}
\begin{figure}[h]
\centering
\includegraphics[width = 0.35\linewidth]{../gallery/bananas.pdf}
\caption{Árbol de prefijos con intervalos SA de la palabra ``bananas\$''.}
\label{fig:arbol3}
\end{figure}

\begin{figure}[h]
\centering
\includegraphics[width = \linewidth]{../gallery/gen.pdf}
\caption{Árbol de prefijos con intervalos SA de la palabra ``cgatagtcgggatggttgccg\$''.}
\label{fig:arbol4}
\end{figure}
\clearpage
\section{Anejo}

\lstinputlisting[
	frame=single,
	mathescape=false,
	basicstyle=\linespread{1}\ttfamily,
	caption={\protect\python inexrecur\_clean.py},
]{../inexrecur_clean.py}

\lstinputlisting[
	frame=single,
	mathescape=false,
	basicstyle=\linespread{1}\ttfamily,
	caption={\protect\R inexrecur\_clean.R},
]{../inexrecur_clean.R}

\lstinputlisting[
	firstline=27,
	lastline=85,
	frame=single,
	mathescape=false,
	basicstyle=\linespread{1}\ttfamily,
	caption={\protect\C inexrecur\_clean.c},
]{../inexrecur_clean.c}

\lstinputlisting[
	firstline=85,
	% lastline=115,
	firstnumber=59,
	frame=single,
	mathescape=false,
	basicstyle=\linespread{1}\ttfamily,
	caption={\C inexrecur\_clean.c (cont.)},
]{../inexrecur_clean.c}
\vfill

\bibliographystyle{IEEEtran}
\bibliography{citations}
\end{document}
